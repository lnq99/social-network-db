\chapter*{Введение}
\addcontentsline{toc}{chapter}{Введение}


Социальная сеть, также известная как виртуальная социальная сеть (англ. Social network), - это служба, которая объединяет участников со схожими интересами в Интернете для различных целей независимо от места и времени. Социальная сеть имеет такие функции, как чат, просмотр фотографий, голосовой чат, обмен файлами, блог и редактирование. Сеть полностью меняет способы связи пользователей сети друг с другом и становится неотъемлемой частью повседневной жизни сотен миллионов участников по всему миру. У этих сервисов есть много способов для участников найти друзей и партнеров: на основе группа (например, название школы или города), на основе личной информации (например, адрес электронной почты или имя) или на основе личных интересов (например, спорт, фильмы, книги или музыка), области интересов : бизнес, купля-продажа. В настоящее время в мире существуют сотни различных социальных сетей, известных как Facebook (2,85 миллиарда активных пользователей), Youtube, Whatsapp, Instagram, TikTok,... и в России - VK.
\\

Цель работы: Разработать прототип социальной сети. Спроектировать и реализовать базу данных приложения. Программа предоставляет базовые функции, такие как регистрация, авторизация, публикация, комментирование, реагирование, поиск людей, изменение отношений (отправка, принятие запроса на добавление в друзья, блокировка людей).
\\

Итак, подзадачи:

\begin{itemize}
    \setlength{\itemsep}{0em}
    \item Формализовать задание
    \item Проанализировать и выбрать  СУБД
    \item Выбрать язык программирования и среду разработки
    \item Выполняется проектирование базы данных (реализуются объекты базы данных, реализуются запросы к базе данных)
    \item Написать генератор данных, моделирующий социальную сеть
    \item Выполняется разработка приложения
    \item Описать презентационного уровня приложения
    % \item Исследовать 
\end{itemize}
