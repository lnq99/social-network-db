\chapter{Аналитический раздел}
\label{cha:analysis}

В данном разделе будут рассмотрены общие сведения о БД и СУБД,
некоторые модели БД с их типичными СУБД и использованные технологии для ПО.

\section{Формализация задачи}

В соответствии с техническим заданием на курсовой проект необходимо разработать
веб-приложение, в котором пользователь может зарегистрироваться, войти в систему, подружиться,
опубликовать, комментировать, реагировать, получать базовые уведомления.

\subsection*{Требования к программу}

Ниже представлены функции по приоритету:

\begin{itemize}
  \setlength{\itemsep}{0em}
  \item Необходимые: аутентификация, авторизация, опубликовать запись (с фото) и удалить пост,
  комментировать, ответ на комментарий (вложенный древовидный комментарий),
  реагировать на записи, у пользователя есть страница профиля и страница с фотографией,
  поиск пользователя по имени.
  \item Желательные: аутентификация через jwt, кэшировать некоторую основную информацию на стороне клиента.
  \item Возможные: темный интерфейс, добавить фильтр для поиска.
\end{itemize}


\section{Общие сведения о БД и СУБД}

База данных - совокупность взаимосвязанных данных некоторой предметной области, хранимых в памяти ЭВМ и
организованных таким образом, что эти данные могут быть использованы для решения многих задач многими пользователями.
\\

Основные требования к организации данных:

\begin{itemize}
  \setlength{\itemsep}{0em}
  \item Неизбыточность данных.
  \item Совместное использование данных многими пользователями.
  \item Эффективность доступа к БД.
  \item Целостность данных.
  \item Безопасность данных.
  \item Восстановление данных после программных и аппаратных сбоев.
  \item Независимость данных от прикладных программ.
\end{itemize}

Система управления базами данных (СУБД) - приложение, обеспечивающее создание, хранение, обновление и поиск информации в базах данных.


\section{Модели данных и СУБД}

По модели данных СУБД можно разделить на несколько типов: инвертированные списки, иерархичекие, сетевые, реляционные, ...
Ниже приведены некоторые модели, которые, на мой взгляд, могут подойти для управления данными в социальных сетях.


\subsection*{Реляционная модель данных и PostgreSQL}

Реляционная модель данных включает следующие компоненты:
структурный, целостностный, манипуляционный.

Основными понятиями структурной части реляционной модели являются тип данных, домен, атрибут, схема
отношения, схема базы данных, кортеж, отношение, потенциальный, первичный и альтернативные ключи, реляционная
база данных.

В целостностной части реляционной модели фиксируются два базовых требования целостности, это целостность сущностей и ссылочная целостность.

Манипуляционная часть реляционной модели описывает два эквивалентных способа манипулирования реляционными
данными – реляционную алгебру и реляционное исчисление.
\\

PostgreSQL - объектно-реляционная система управления базами данных.
Среди СУБД, таких как Oracle, MySQL, MSSQL, PostgreSQL, ... Я выбираю Postgres, потому что это бесплатное программное обеспечение, и у меня также есть опыт работы с ним.
Postgresql основан на SQL, ACID-совместимый, но поддерживает различные функции NoSQL (array, json, bjson).
Базы данных SQL по большей части масштабируются по вертикали.
Высокоструктурированная природа реляционных баз данных также является их недостатком, поскольку они жестко привязаны к определенной природе столбцов в своих таблицах.


\subsection*{Графовая модель данных и Neo4j}

К основным понятиям сетевой модели БД относятся: элемент (узел), связь. Узел - это совокупность атрибутов данных,
описывающих некоторый объект. Сетевые БД могут быть представлены в виде графа.
Графовая база данных - разновидность баз данных с реализацией сетевой модели в виде графа и его обобщений. 
\\

Neo4j - графовая система управления базами данных с открытым исходным кодом.
Данные хранит в собственном формате, специализированно приспособленном для представления графовой информации, такой подход в сравнении с моделированием графовой базы данных средствами реляционной СУБД позволяет применять дополнительную оптимизацию в случае данных с более сложной структурой.

Когда отношения становятся важными данными и содержат много информации, графовая база данных - достойный выбор. Например, графовая база данных работает быстро при поиске отношений типа друзья друга.


\subsection*{Документоориентированная модель данных и MongoDB}

Документоориентированная СУБД (англ. document-oriented) - СУБД, специально предназначенная для хранения иерархических структур данных (документов) и обычно реализуемая с помощью подхода NoSQL.
В основе документоориентированных СУБД лежат документные хранилища, имеющие структуру дерева (иногда леса).
\\

MongoDB - документоориентированная система управления базами данных, не требующая описания схемы таблиц.
Считается одним из классических примеров NoSQL-систем, использует JSON-подобные документы и схему базы данных.

MongoDB обеспечивает хорошую производительность запросов, но это не оптимальное решение, если приложению требуется часто обновлять базу данных. Он имеет распределенную архитектуру, поэтому обладает хорошей масштабируемостью.

% \begin{figure}[ht]
%   \centering
%   \includegraphics[width=0.35\textwidth]{img/warnock_1.png}
%   \hspace{1cm}
%   \includegraphics[width=0.35\textwidth]{img/warnock_2.png}
%   \caption{Алгоритм Варнока}
% \end{figure}


\subsection*{Вывод}

В общем, как я знаю, Neo4j хорош, когда отношения рассматриваются как важный тип данных и нуждаются в глубоком обходном отношении.
Mongodb хорош для выполнения запросов и обладает хорошей гибкостью, масштабируемостью но он не очень быстро для соединения или записи данных.
С целью создания очень простой социальной сети, без сложных взаимоотношений и без больших требований к масштабируемости, тогда Postgresql лучше всего подходит.
Кроме того, у меня также есть опыт работы с Postgresql, поэтому в рамках курсовой работы я считаю, что это правильный выбор.


% \section{Технология для веб-приложений}


\section{Вывод}

В этом разделе было проанализированы некоторые модели БД с их типичными СУБД и пришли к решению использовать Postgresql.
% Также кратко описаны Gin - http веб-фреймворк и Vuejs прогрессивный фреймворк для создания пользовательских интерфейсов.